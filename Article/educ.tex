\RequirePackage[l2tabu, orthodox]{nag}
\documentclass[pagesize, twoside=off, bibliography=totoc, DIV=calc, fontsize=12pt, a4paper, french]{scrartcl}
%Permits to copy eg x ⪰ y ⇔ v(x) ≥ v(y) from PDF to unicode data, and to search. From pdfTeX users manual. See https://tex.stackexchange.com/posts/comments/1203887.
	\input glyphtounicode
	\pdfgentounicode=1
%Latin Modern has more glyphs than Computer Modern, such as diacritical characters. fntguide commands to load the font before fontenc, to prevent default loading of cmr.
	\usepackage{lmodern}
%Encode resulting accented characters correctly in resulting PDF, permits copy from PDF.
	\usepackage[T1]{fontenc}
%UTF8 seems to be the default in recent TeX installations, but not all, see https://tex.stackexchange.com/a/370280.
	\usepackage[utf8]{inputenc}
%Provides \newunicodechar for easy definition of supplementary UTF8 characters such as → or ≤ for use in source code.
	\usepackage{newunicodechar}
%Text Companion fonts, much used together with CM-like fonts. Provides \texteuro and commands for text mode characters such as \textminus, \textrightarrow, \textlbrackdbl.
	\usepackage{textcomp}
%For \ding, that is, ✓ and ✗, thanks to https://tex.stackexchange.com/a/42620
	%\usepackage{pifont}
%St Mary’s Road symbol font, used for ⟦ = \llbracket. The \SetSymbolFont command avoids spurious warnings, but also some valid ones, see https://tex.stackexchange.com/a/106719/.
	%\usepackage{stmaryrd}\SetSymbolFont{stmry}{bold}{U}{stmry}{m}{n}
%Solves bug in lmodern, https://tex.stackexchange.com/a/261188; probably useful only for unusually big font sizes; and probably better to use exscale instead. Note that the authors of exscale write against this trick.
	%\DeclareFontShape{OMX}{cmex}{m}{n}{
		%<-7.5> cmex7
		%<7.5-8.5> cmex8
		%<8.5-9.5> cmex9
		%<9.5-> cmex10
	%}{}
	%\SetSymbolFont{largesymbols}{normal}{OMX}{cmex}{m}{n}
%More symbols (such as \sum) available in bold version, see https://github.com/latex3/latex2e/issues/71. In article mode (but not in presentation mode), also hides some potentially useful warnings such as when using $\bm{\llbracket}$, see stmaryrd in this document (not sure why).
	\DeclareFontShape{OMX}{cmex}{bx}{n}{%
	   <->sfixed*cmexb10%
	   }{}
	\SetSymbolFont{largesymbols}{bold}{OMX}{cmex}{bx}{n}
%For small caps also in italics, see https://tex.stackexchange.com/questions/32942/italic-shape-needed-in-small-caps-fonts, https://tex.stackexchange.com/questions/284338/italic-small-caps-not-working.
	\usepackage{slantsc}
	\AtBeginDocument{%
		%“Since nearly no font family will contain real italic small caps variants, the best approach is to substitute them by slanted variants.” -- slantsc doc
		%\DeclareFontShape{T1}{lmr}{m}{scit}{<->ssub*lmr/m/scsl}{}%
		%There’s no bold small caps in Latin Modern, we switch to Computer Modern for bold small caps, see https://tex.stackexchange.com/a/22241
		%\DeclareFontShape{T1}{lmr}{bx}{sc}{<->ssub*cmr/bx/sc}{}%
		%\DeclareFontShape{T1}{lmr}{bx}{scit}{<->ssub*cmr/bx/scsl}{}%
	}
%Warn about missing characters.
	\tracinglostchars=2
%Nicer tables: provides \toprule, \midrule, \bottomrule.
	%\usepackage{booktabs}
%For new column type X which stretches; can be used together with booktabs, see https://tex.stackexchange.com/a/97137. “tabularx modifies the widths of the columns, whereas tabular* modifies the widths of the inter-column spaces.” Loads array.
	%\usepackage{tabularx}
%math-mode version of "l" column type. Requires \usepackage{array}.
	%\usepackage{array}
	%\newcolumntype{L}{>{$}l<{$}}
%Provides \xpretocmd and loads etoolbox which provides \apptocmd, \patchcmd, \newtoggle… Also loads xparse, which provides \NewDocumentCommand and similar commands intended as replacement of \newcommand in LaTeX3 for defining commands (see https://tex.stackexchange.com/q/98152 and https://github.com/latex3/latex2e/issues/89).
	\usepackage{xpatch}
%for \nexists and because it is a basic package nowadays, see https://tex.stackexchange.com/q/539592/.
	\usepackage{amssymb}
%loads and fixes some bugs in amsmath (a basic, mandatory package nowadays, see Grätzer, More Math Into LaTeX) and provides \DeclarePairedDelimiter. I recommend \begin{equation}, which allows numbering, rather than \[ (and $$ should be avoided), see https://tex.stackexchange.com/questions/503. Relatedly, do not use the displaymath environment: use equation. Do not use the eqnarray environment: use align. This improves spacing. (See l2tabu or amsldoc.)
	\usepackage{mathtools}
%Package frenchb asks to load natbib before babel-french. Package hyperref asks to load natbib before hyperref.
	\usepackage{natbib}

  \newtoggle{LCpres}
  \newtoggle{LCart}
  \newtoggle{LCposter}
  \newtoggle{LCfrench}
  \makeatletter
  \@ifclassloaded{beamer}{
    \toggletrue{LCpres}
    \togglefalse{LCart}
    \togglefalse{LCposter}
    \@ifclasswith{beamer}{french}{\toggletrue{LCfrench}}{\togglefalse{LCfrench}}
    \wlog{Presentation mode}
    }{
    \@ifclassloaded{tikzposter}{
      \toggletrue{LCposter}
			\togglefalse{LCpres}
			\togglefalse{LCart}
      \toggletrue{LCfrench}
			\wlog{Poster mode}
    }{
      \toggletrue{LCart}
      \togglefalse{LCpres}
      \togglefalse{LCposter}
      \@ifclasswith{scrartcl}{french}{\toggletrue{LCfrench}}{\togglefalse{LCfrench}}
			\wlog{Article mode}
		}
	}
	\makeatother%

%Quick recap about beamer’s overlay modes using \command<spec>{text}: \uncover always takes space and \only only takes space when shown; I recommend not using \onslide. Use \zeroboxonly for never taking space.
	\iftoggle{LCpres}{
		\NewDocumentCommand{\zeroboxonly}{mm}{\only<#1>{\makebox[0pt]{#2}}}
	}{
	}
%Language options ([french, english]) should be on the document level (last is main); except with tikzposter: put [french, english] options next to \usepackage{babel} to avoid warning (not yet implemented here). beamer uses the \translate command for the appendix: omitting babel results in a warning, see https://github.com/josephwright/beamer/issues/449. Babel also seems required for \refname.
\ifboolexpr{ togl {LCpres} or togl {LCfrench} }{
		\usepackage{babel}
%https://tex.stackexchange.com/questions/162137/loading-microtype-before-or-after-the-font
		\usepackage[babel]{microtype}
	}{
		\usepackage{microtype}
	}
	\iftoggle{LCfrench}{
		\frenchbsetup{AutoSpacePunctuation=false}
	}{
	}
%https://ctan.org/pkg/amsmath recommends ntheorem, which supersedes amsthm, which corrects the spacing of proclamations and allows for theoremstyle, but I decided to switch to amsthm with thmtools (mentioned in amsthm doc) because ntheorem “seems essentially unmaintaned and has severe problems”, see https://tex.stackexchange.com/q/535950. Must be loaded after amsmath (from amsthm doc).
		\usepackage{amsthm}
		\usepackage{thmtools}
%listings (1.7) does not allow multi-byte encodings. listingsutf8 works around this only for characters that can be represented in a known one-byte encoding and only for \lstinputlisting. Other workarounds: use literate mechanism; or escape to LaTeX (but breaks alignment).
	%\usepackage{listings}
	%\lstset{tabsize=2, basicstyle=\ttfamily, escapechar=§, literate={é}{{\'e}}1}
%I favor acro over acronym because the former is more recently updated (2018 VS 2015 at time of writing); has a longer user manual (about 40 pages VS 6 pages if not counting the example and implementation parts); has a command for capitalization; and acronym suffers a nasty bug when ac used in section, see https://tex.stackexchange.com/q/103483 (though this might be the fault of the silence package and might be solved in more recent versions, I do not know) and from a bug when used with cleveref, see https://tex.stackexchange.com/q/71364. However, it seems to suffer from performance issues. I opened an issue about loading it making compilation time (one pass on this template) go from 0.6 to 1.4 seconds at https://bitbucket.org/cgnieder/acro/issues/115, which seems to have now disappeared, and https://github.com/cgnieder/acro/issues/205 is another one.
	% \usepackage{acro}
	%“All options of acro that have not been mentioned in section 4.1 have to be set up… with… \acsetup{…}” -- acro package doc, cited by the Overleaf support (thanks to them!)
	% \acsetup{single, use-id-as-short}
	% \DeclareAcronym{AMCD}{long={Aide Multicritère à la Décision}}
\DeclareAcronym{AHP}{long={Analytic Hierarchy Process}}
\DeclareAcronym{AR}{long={Argumentative Recommender}}
\DeclareAcronym{DA}{long={Decision Analysis}}
\DeclareAcronym{DJ}{long={Deliberated Judgment}}
\DeclareAcronym{DM}{long={Decision Maker}}
\DeclareAcronym{DP}{long={Deliberated Preference}}
\DeclareAcronym{MAVT}{long={Multiple Attribute Value Theory}}
\DeclareAcronym{MCDA}{long={Multicriteria Decision Aid}}
\DeclareAcronym{MIP}{long={Mixed Integer Program}}
\DeclareAcronym{SCR}{long={Social Choice Rule}}
\DeclareAcronym{SEU}{long={Subjective Expected Utility}}


\iftoggle{LCpres}{
	%I favor fmtcount over nth because it is loaded by datetime anyway; and fmtcount warns about possible conflicts when loaded after nth (“\ordinal already defined use \FCordinal instead”). See also https://english.stackexchange.com/questions/93008.
	\usepackage{fmtcount}
	%For nice input of date of presentation. Must be loaded after the babel package. Has possible problems with srcletter: https://golatex.de/verwendung-von-babel-und-datetime-in-scrlttr2-schlaegt-fehlt-t14779.html.
	\usepackage[nodayofweek]{datetime}
}{
}
%For presentations, Beamer implicitely uses the pdfusetitle option. autonum doc mandates option hypertexnames=false. I want to highlight links only if necessary for the reader to recognize it as a link, to reduce distraction. In presentations, this is already taken care of by beamer (https://tex.stackexchange.com/a/262014). If using colorlinks=true in a presentation, see https://tex.stackexchange.com/q/203056. Crashes the first compilation with tikzposter, just compile again and the problem disappears, see https://tex.stackexchange.com/q/254257.
\makeatletter
\iftoggle{LCpres}{
	\usepackage{hyperref}
}{% Used to have linkbordercolor={1 1 1}, citebordercolor={1 1 1}, urlbordercolor={1 1 1} instead of pdfborder={0 0 0} but this creates visual glitches in some PDF viewers.
	\usepackage[hypertexnames=false, pdfusetitle, pdfborder={0 0 0}, breaklinks]{hyperref}
	%https://tex.stackexchange.com/a/466235
	\pdfstringdefDisableCommands{%
		\let\thanks\@gobble
	}
}
\makeatother
%urlbordercolor is used both for \url and \doi, which I think shouldn’t be colored, and for \href, thus might want to color manually when required. Requires xcolor.
	\NewDocumentCommand{\hrefblue}{mm}{\textcolor{blue}{\href{#1}{#2}}}
%hyperref doc says: “Package bookmark replaces hyperref’s bookmark organization by a new algorithm (...) Therefore I recommend using this package”.
	\usepackage{bookmark}
%Need to invoke hyperref explicitly to link to line numbers: \hyperlink{lintarget:mylinelabel}{\ref*{lin:mylinelabel}}, with \ref* to disable automatic link. Also see https://tex.stackexchange.com/q/428656 for referencing lines from another document.
	%\usepackage{lineno}
	%\NewDocumentCommand{\llabel}{m}{\hypertarget{lintarget:#1}{}\linelabel{lin:#1}}
	%\setlength\linenumbersep{9mm}
%For complex authors blocks. Seems like authblk wants to be later than hyperref, but sooner than silence. See https://tex.stackexchange.com/q/475513 for the patch to hyperref pdfauthor.
	\ExplSyntaxOn
	\seq_new:N \g_oc_hrauthor_seq
	\NewDocumentCommand{\addhrauthor}{m}{
		\seq_gput_right:Nn \g_oc_hrauthor_seq { #1 }
	}
	\NewExpandableDocumentCommand{\hrauthor}{}{
		\seq_use:Nn \g_oc_hrauthor_seq {,~}
	}
	\ExplSyntaxOff
	{
		\catcode`#=11\relax
		\gdef\fixauthor{\xpretocmd{\author}{\addhrauthor{#2}}{}{}}%
	}
	\iftoggle{LCart}{
		\usepackage{authblk}
		\renewcommand\Affilfont{\small}
		\fixauthor
		\AtBeginDocument{
		    \hypersetup{pdfauthor={\hrauthor}}
		}
	}{
	}
%I do not use floatrow, because it requires an ugly hack for proper functioning with KOMA script (see scrhack doc). Instead, the following command centers all floats (using \centering, as the center environment adds space, http://texblog.net/latex-archive/layout/center-centering/), and I manually place my table captions above and figure captions below their contents (https://tex.stackexchange.com/a/3253).
	\makeatletter
	\g@addto@macro\@floatboxreset\centering
	\makeatother
%Permits to customize enumeration display and references
	%\nottoggle{LCpres}{
		%\usepackage{enumitem} %follow list environments by a string to customize enumeration, example: \begin{description}[itemindent=8em, labelwidth=!] or \begin{enumerate}[label=({\roman*}), ref={\roman*}].
	%}{
	%}
%Provides \Centering, \RaggedLeft, and \RaggedRight and environments Center, FlushLeft, and FlushRight, which allow hyphenation. Also does not suffer from bug https://tex.stackexchange.com/a/35350/. With tikzposter, seems to cause 1=1 to be printed in the middle of the poster.
	%\usepackage{ragged2e}
%To typeset units by closely following the “official” rules.
	%\usepackage{siunitx}
%Loads url, provides better long line handling (https://tex.stackexchange.com/q/716301)
  \usepackage{xurl}
%Turns the doi provided by some bibliography styles into URLs.
	\usepackage{doi}
%Makes sure upper case greek letters are italic as well.
	\usepackage{fixmath}
%Provides \mathbb; obsoletes latexsym (see http://tug.ctan.org/macros/latex/base/latexsym.dtx). Relatedly, \usepackage{eucal} to change the mathcal font and \usepackage[mathscr]{eucal} (apparently equivalent to \usepackage[mathscr]{euscript}) to supplement \mathcal with \mathscr. This last option is not very useful as both fonts are similar, and the intent of the authors of eucal was to provide a replacement to mathcal (see doc euscript). Also provides \mathfrak for supplementary letters.
	\usepackage{amsfonts}
%Provides a beautiful (IMHO) \mathscr and really different than \mathcal, for supplementary uppercase letters. But there is no bold version. Alternative: mathrsfs (more slanted), but when used with tikzposter, it warns about size substitution, see https://tex.stackexchange.com/q/495167.
	\usepackage[scr]{rsfso}
%Multiple means to produce bold math: \mathbf, \boldmath (defined to be \mathversion{bold}, see fntguide), \pmb, \boldsymbol (all legacy, from LaTeX base and AMS), \bm (the most recommended one), \mathbold from package fixmath (I don’t see its advantage over \boldsymbol).
%“The \boldsymbol command is obtained preferably by using the bm package, which provides a newer, more powerful version than the one provided by the amsmath package. Generally speaking, it is ill-advised to apply \boldsymbol to more than one symbol at a time.” — AMS Short math guide. “If no bold font appears to be available for a particular symbol, \bm will use ‘poor man’s bold’” — bm. It is “best to load the package after any packages that define new symbol fonts” – bm. bm defines \boldsymbol as synonym to \bm. \boldmath accesses the correct font if it exists; it is used by \bm when appropriate. See https://tex.stackexchange.com/a/10643 and https://github.com/latex3/latex2e/issues/71 for some difficulties with \bm.
	\usepackage{bm}
	\nottoggle{LCpres}{
	%Provides \cref. Unfortunately, cref fails when the language is French and referring to a label whose name contains a colon (https://tex.stackexchange.com/q/83798). Use \cref{sec\string:intro} to work around this. cleveref should go “laster” than hyperref.  Cleveref gets disabled in presentation mode as it do not seem to go well with Beamer, see https://github.com/josephwright/beamer/issues/750.
		\usepackage[capitalise, nameinlink]{cleveref}
    \crefformat{appendix}{l’#2annexe~#1#3}
    }{
	}
	\nottoggle{LCposter}{
	%Equations get numbers iff they are referenced. Loading order should be “amsmath → hyperref → cleveref → autonum”, according to autonum doc. Use this in preference to the showonlyrefs option from mathtools, see https://tex.stackexchange.com/q/459918 and autonum doc. See https://tex.stackexchange.com/a/285953 for the etex line (maybe not needed any more? TODO). Also, autonum loads the outdated package etextools, which makes it difficult to coexist with csvsimple (https://tex.stackexchange.com/q/213267/). Incompatible with my version of tikzposter (produces “! Improper \prevdepth”). This removes the starred versions, such as equation*. Unfortunately, this prevents using \qedhere in an equation ending a proof, see https://tex.stackexchange.com/q/133358/.
		\expandafter\def\csname ver@etex.sty\endcsname{3000/12/31}\let\globcount\newcount
		\usepackage{autonum}
	}{
	}
%Also loaded by tikz.
	%\usepackage{xcolor}
%For comments highlight, \hl
	%\usepackage{soul}
\iftoggle{LCpres}{
	\usepackage{tikz}
	%\usetikzlibrary{babel, matrix, fit, plotmarks, calc, trees, shapes.geometric, positioning, plothandlers, arrows, shapes.multipart}
}{
}
%Vizualization, on top of TikZ
	%\usepackage{pgfplots}
	%\pgfplotsset{compat=1.14}
\usepackage{graphicx}
	\graphicspath{{graphics/}}

%Provides \printlength{length}, useful for debugging.
	%\usepackage{printlen}
	%\uselengthunit{mm}

\iftoggle{LCpres}{
	\usepackage{appendixnumberbeamer}
	%I have yet to see anyone actually use these navigation symbols; let’s disable them
	\setbeamertemplate{navigation symbols}{} 
	\usepackage{preamble/beamerthemeParisFrance}
	\setcounter{tocdepth}{10}
}{
}

%Requires package xcolor and soul.
%\definecolor{ao(english)}{rgb}{0.0, 0.5, 0.0}
%I observed that two braces around \small are required to make it local.
\NewDocumentCommand{\commentOC}{O{}m}{\ifx&#1&\else\hl{#1} \fi\textcolor{blue}{{\small$\big[$OC: #2$\big]$}}}
%Requires package babel and option [french]. According to babel doc, need two braces around \selectlanguage to make the changes really local.
\NewDocumentCommand{\commentOCf}{O{}m}{\ifx&#1&\else\hl{#1} \fi\textcolor{blue}{{\small\selectlanguage{french}$\big[$OC : #2$\big]$}}}
\NewDocumentCommand{\commentYM}{O{}m}{\ifx&#1&\else\hl{#1} \fi\textcolor{red}{{\small$\big[$YM: #2$\big]$}}}

\bibliographystyle{abbrvnat}
%https://tex.stackexchange.com/questions/125690
\NewDocumentCommand{\possessivecite}{O{}m}{\citeauthor{#2}’s \citeyearpar[#1]{#2}}
\NewDocumentCommand{\Possessivecite}{O{}m}{\Citeauthor{#2}’s \citeyearpar[#1]{#2}}
\NewDocumentCommand{\email}{m}{\href{mailto:#1}{\nolinkurl{#1}}}

\iftoggle{LCart}{
	\declaretheorem{theorem}
	\declaretheorem{lemma}
	\declaretheorem{proposition}
	\declaretheorem{conjecture}
	\declaretheorem{corollary}
	\declaretheorem{axiom}
	\crefname{axiom}{Axiom}{Axioms}
	\declaretheorem{condition}
	\declaretheorem{property}
	\declaretheorem{procedure}
%Possibly: qed=\scalebox{1.2}{$\hexagon$} (and \usepackage{wasysym}, after amssymb). But probably not useful as most definitions are one § only. Might also consider: circledast, circledcirc, circleddash; black triangle or black circle; diamond…
	\declaretheorem[style=definition]{definition}
	\declaretheorem[style=remark, qed=$\triangle$]{example}
	\declaretheorem[style=remark, qed=$\triangle$]{remark}
}

%https://tex.stackexchange.com/a/467188, https://tex.stackexchange.com/a/36088 - uncomment if one of those symbols is used.
%\DeclareFontFamily{U} {MnSymbolD}{}
%\DeclareFontShape{U}{MnSymbolD}{m}{n}{
%  <-6> MnSymbolD5
%  <6-7> MnSymbolD6
%  <7-8> MnSymbolD7
%  <8-9> MnSymbolD8
%  <9-10> MnSymbolD9
%  <10-12> MnSymbolD10
%  <12-> MnSymbolD12}{}
%\DeclareFontShape{U}{MnSymbolD}{b}{n}{
%  <-6> MnSymbolD-Bold5
%  <6-7> MnSymbolD-Bold6
%  <7-8> MnSymbolD-Bold7
%  <8-9> MnSymbolD-Bold8
%  <9-10> MnSymbolD-Bold9
%  <10-12> MnSymbolD-Bold10
%  <12-> MnSymbolD-Bold12}{}
%\DeclareSymbolFont{MnSyD} {U} {MnSymbolD}{m}{n}
%\DeclareMathSymbol{\ntriplesim}{\mathrel}{MnSyD}{126}
%\DeclareMathSymbol{\nlessgtr}{\mathrel}{MnSyD}{192}
%\DeclareMathSymbol{\ngtrless}{\mathrel}{MnSyD}{193}
%\DeclareMathSymbol{\nlesseqgtr}{\mathrel}{MnSyD}{194}
%\DeclareMathSymbol{\ngtreqless}{\mathrel}{MnSyD}{195}
%\DeclareMathSymbol{\nlesseqgtrslant}{\mathrel}{MnSyD}{198}
%\DeclareMathSymbol{\ngtreqlessslant}{\mathrel}{MnSyD}{199}
%\DeclareMathSymbol{\npreccurlyeq}{\mathrel}{MnSyD}{228}
%\DeclareMathSymbol{\nsucccurlyeq}{\mathrel}{MnSyD}{229}
%\DeclareFontFamily{U} {MnSymbolA}{}
%\DeclareFontShape{U}{MnSymbolA}{m}{n}{
%  <-6> MnSymbolA5
%  <6-7> MnSymbolA6
%  <7-8> MnSymbolA7
%  <8-9> MnSymbolA8
%  <9-10> MnSymbolA9
%  <10-12> MnSymbolA10
%  <12-> MnSymbolA12}{}
%\DeclareFontShape{U}{MnSymbolA}{b}{n}{
%  <-6> MnSymbolA-Bold5
%  <6-7> MnSymbolA-Bold6
%  <7-8> MnSymbolA-Bold7
%  <8-9> MnSymbolA-Bold8
%  <9-10> MnSymbolA-Bold9
%  <10-12> MnSymbolA-Bold10
%  <12-> MnSymbolA-Bold12}{}
%\DeclareSymbolFont{MnSyA} {U} {MnSymbolA}{m}{n}
%%Rightwards wave arrow: ↝. Alternative: \rightsquigarrow from amssymb, but it’s uglier
%\DeclareMathSymbol{\rightlsquigarrow}{\mathrel}{MnSyA}{160}

%03B1 Greek Small Letter Alpha
\newunicodechar{α}{\alpha}
\newunicodechar{β}{\beta}
%03B3 Greek Small Letter Gamma
\newunicodechar{γ}{\gamma}
%03B4 Greek Small Letter Delta
\newunicodechar{δ}{\delta}
\newunicodechar{ε}{\epsilon}
\newunicodechar{λ}{\lambda}
\newunicodechar{μ}{\mu}
\newunicodechar{ν}{\nu}
\newunicodechar{π}{\pi}
\newunicodechar{τ}{\tau}
%03C6 Greek Small Letter Phi (“the ordinary Greek letter, showing considerable glyph variation; in mathematical contexts, the loopy glyph is preferred, to contrast with 03D5 φ” – The Unicode Standard 13.0, https://www.unicode.org/charts/PDF/U0370.pdf)
\newunicodechar{φ}{\phi}
%03D5 Greek Phi symbol (“used as a technical symbol, with a stroked glyph” – The Unicode Standard 13.0, https://www.unicode.org/charts/PDF/U0370.pdf)
\newunicodechar{ϕ}{\varphi}
%03A6 Φ Greek Capital Letter Phi
\newunicodechar{Φ}{\Phi}
%03A8 Φ Greek Capital Letter Psi
\newunicodechar{Ψ}{\Psi}
%2115 Double-Struck Capital N
\newunicodechar{ℕ}{\mathbb{N}}
%211A Double-Struck Capital Q
\newunicodechar{ℚ}{\mathbb{Q}}
%211D Double-Struck Capital R
\newunicodechar{ℝ}{\mathbb{R}}
\newunicodechar{ℤ}{\mathbb{Z}}
%2194 
\newunicodechar{↔}{\leftrightarrow}
%21CF Rightwards Double Arrow with Stroke
\newunicodechar{⇏}{\nRightarrow}
%21D0 Leftwards Double Arrow
\newunicodechar{⇐}{\ensuremath{\Leftarrow}}
%21D2 Rightwards Double Arrow
\newunicodechar{⇒}{\ensuremath{\Rightarrow}}
%21D4 Left Right Double Arrow
\newunicodechar{⇔}{\Leftrightarrow}
%21DD Rightwards Squiggle Arrow
\newunicodechar{⇝}{\rightsquigarrow}
%2205 Empty Set
\newunicodechar{∅}{\emptyset}
%2212 Minus Sign
\newunicodechar{−}{\ifmmode{-}\else\textminus\fi}
%221E Infinity
\newunicodechar{∞}{\infty}
%2227 Logical And
\newunicodechar{∧}{\land}
%2228 Logical Or
\newunicodechar{∨}{\lor}
%2229 Intersection
\newunicodechar{∩}{\cap}
%222A Union
\newunicodechar{∪}{\cup}
%2260 Not Equal To (handy also as text in informal writing)
\newunicodechar{≠}{\ensuremath{\neq}}
%2264 Less-Than or Equal To
\newunicodechar{≤}{\leq}
%2265 Greater-Than or Equal To
\newunicodechar{≥}{\geq}
%2270 Neither Less-Than nor Equal To
\newunicodechar{≰}{\nleq}
%2271 Neither Greater-Than nor Equal To
\newunicodechar{≱}{\ngeq}
%2272 Less-Than or Equivalent To
\newunicodechar{≲}{\lesssim}
%2273 Greater-Than or Equivalent To
\newunicodechar{≳}{\gtrsim}
%2274 Neither Less-Than nor Equivalent To – also, from MnSymbol: \nprecsim, a more exact match to the Unicode symbol; and \npreccurlyeq, too small
\newunicodechar{≴}{\not\preccurlyeq}
%2275 Neither Greater-Than nor Equivalent To
\newunicodechar{≵}{\not\succcurlyeq}
%2279 Neither Greater-Than nor Less-Than – requires MnSymbol; also \nlessgtr from txfonts/pxfonts, \ngtreqless from MnSymbol (but much higher), \ngtrless from MnSymbol (a more exact match to the Unicode symbol); for incomparability (not matching this Unicode symbol), may also consider \ntriplesim from MnSymbol,\nparallelslant from fourier, \between from mathabx, or ⋈
\newunicodechar{≹}{\ngtreqlessslant}
%227A Precedes
\newunicodechar{≺}{\prec}
%227B Succeeds
\newunicodechar{≻}{\succ}
%227C Precedes or Equal To
\newunicodechar{≼}{\preccurlyeq}
%227D Succeeds or Equal To
\newunicodechar{≽}{\succcurlyeq}
%227E Precedes or Equivalent To
\newunicodechar{≾}{\precsim}
%227F Succeeds or Equivalent To
\newunicodechar{≿}{\succsim}
%2280 Does Not Precede
\newunicodechar{⊀}{\nprec}
%2281 Does Not Succeed
\newunicodechar{⊁}{\nsucc}
%2286
\newunicodechar{⊆}{\subseteq}
%22A2
\newunicodechar{⊢}{\vdash}
%22A5 Up Tack
\newunicodechar{⊥}{\bot}
%22AC
\newunicodechar{⊬}{\nvdash}
%22B2 Normal Subgroup Of – using \vartriangleleft from amsfonts, which goes well with \trianglelefteq, \ntriangleright, and so on, also from amsfonts; another possibility is \lhd from latexsym, which seems visually equivalent to \vartriangleleft from amsfonts; latexsym also has ⊴=\unlhd, but doesn’t have some other related symbols. Other related symbols: \triangleleft from latesym package is too small; fdsymbol provides \triangleleft=\medtriangleleft and \vartriangleleft=\smalltriangleleft; MnSymbol provides \medtriangleleft and \vartriangleleft=\lessclosed=\lhd which are smaller than \vartriangleleft from amsfont; \vartriangleleft from mathabx (p. 67), looks different (wider); also \vartriangleleft from boisik (p. 69) looks still different; \vartriangleleft=\lhd from stix are smaller. Oddly enough, \triangleright appears as the LMMathItalic12-Regular font whereas \rhd appears as LASY10 and \vartriangleright appears as MSAM10 in the resulting PDF.
\newunicodechar{⊲}{\vartriangleleft}
%22B3 Contains as Normal Subgroup (also: 25B7 White right-pointing triangle or 25B9 White right-pointing small triangle)
\newunicodechar{⊳}{\vartriangleright}
%22B4 Normal Subgroup of or Equal To
\newunicodechar{⊴}{\trianglelefteq}
%22B5 Contains as Normal Subgroup or Equal To
\newunicodechar{⊵}{\trianglerighteq}
%22C8 Bowtie
\newunicodechar{⋈}{\bowtie}
%22EA Not Normal Subgroup Of
\newunicodechar{⋪}{\ntriangleleft}
%22EB Does Not Contain As Normal Subgroup
\newunicodechar{⋫}{\ntriangleright}
%22EC Not Normal Subgroup of or Equal To
\newunicodechar{⋬}{\ntrianglelefteq}
%22ED Does Not Contain as Normal Subgroup or Equal
\newunicodechar{⋭}{\ntrianglerighteq}
%25A1 White Square
\newunicodechar{□}{\Box}
%2713 Check Mark – amssymb \checkmark; bbding \Checkmark \CheckmarkBold; pifont \ding{51} \ding{52}; dingbat \checkmark; thanks to https://tex.stackexchange.com/a/132785 and https://tex.stackexchange.com/questions/132783#comment299869_132789
\newunicodechar{✓}{\checkmark}
%\newunicodechar{✓}{\ifmmode\text{\ding{51}}\else\ding{51}\fi}
%2717 Ballot X
%\newunicodechar{✗}{\ifmmode\text{\ding{55}}\else\ding{55}\fi}
%27E6 Mathematical Left White Square Bracket – requires stmaryrd (alternative: \text{\textlbrackdbl}, but ugly if used in an italicized text such as a theorem)
\newunicodechar{⟦}{\llbracket}
%27E7 Mathematical Right White Square Bracket
\newunicodechar{⟧}{\rrbracket}
%27FC Long Rightwards Arrow from Bar
\newunicodechar{⟼}{\longmapsto}
%2AB0 Succeeds Above Single-Line Equals Sign
\newunicodechar{⪰}{\succeq}
%301A Left White Square Bracket
\newunicodechar{〚}{\textlbrackdbl}
%301B Right White Square Bracket
\newunicodechar{〛}{\textrbrackdbl}

%Using \DeclareUnicodeCharacter instead of \newunicodechar because the latter warns about the previous definition.
%¬ Not Sign
\DeclareUnicodeCharacter{00AC}{\ifmmode\lnot\else\textlnot\fi}
%®
\DeclareUnicodeCharacter{00AE}{\textsuperscript{\textregistered}}
%00B1 Plus-Minus Sign ±
\DeclareUnicodeCharacter{00B1}{\pm}
%× Multiplication Sign
\DeclareUnicodeCharacter{00D7}{\ifmmode\times\else\texttimes\fi}
%… Horizontal Ellipsis
\DeclareUnicodeCharacter{2026}{\ifmmode\dots\else\textellipsis\fi}
%← Leftwards Arrow
\DeclareUnicodeCharacter{2190}{\ifmmode\leftarrow\else\textleftarrow\fi}
%→ is defined by default as \textrightarrow, which is invalid in math mode. Same thing for the three other commands. 
%→ Rightwards Arrow
\DeclareUnicodeCharacter{2192}{\ifmmode\rightarrow\else\textrightarrow\fi}
%→ Rightwards Double Arrow
\DeclareUnicodeCharacter{21D2}{\Rightarrow}
%Permits to really obtain a straight quote when typing a straight quote; potentially dangerous, see https://tex.stackexchange.com/a/521999, and prevents using double primes (x''). Better then use U+2032 for prime [as recommended in Unicode ® Technical Report #25 UNICODE SUPPORT FOR MATHEMATICS], U+2033 for double prime, U+2034 for triple prime and define the appropriate commands. But arguably that’s for purists.
%\catcode`\'=\active
%\DeclareUnicodeCharacter{0027}{\ifmmode^\prime\else\textquotesingle\fi}

%I find these settings useful in draft mode. Should be removed for final versions.
	%Which line breaks are chosen: accept worse lines, therefore reducing risk of overfull lines. Default = 200.
	\tolerance=2000
	%Accept overfull hbox up to...
		\hfuzz=2cm
	%Reduces verbosity about the bad line breaks.
		\hbadness 5000
	%Reduces verbosity about the underful vboxes.
		\vbadness=1300

\usepackage[top=2.5cm, bottom=2.6cm, left=2.22cm, right=2.22cm]{geometry}

\title{L’influence de l’éducation et des politiques éducatives sur les inégalités de revenu en Amérique Latine}
\author{Ouissem Chikh}
\author{Morgane Le Plomb}
\author{Olivier Cailloux}
\affil{Université Paris-Dauphine, PSL Research University, CNRS, 75016 PARIS, FRANCE}
\hypersetup{
	pdfsubject={},
	pdfkeywords={},
}
% Remplacer (sans appendix)
% ([^\n])\n([^\n])
% par
% $1\n\n$2
\begin{document}
\maketitle
\pagenumbering{roman}
\tableofcontents
\newpage
\setcounter{page}{1}
\pagenumbering{arabic}

\section{Introduction}

La productivité mondiale a énormément progressé au cours des derniers siècles, au point que la plupart des pays, y compris ceux en voie de développement, permettra à une part non négligeable de la population de vivre avec aisance, voire, une très grande aisance, pour une proportion faible de la population. L’éducation a joué historiquement et continue à jouer un rôle dans l’essor d’une grande productivité, clé du dégagement d’un surplus suffisant pour bien vivre. Dans le même temps, face à une inégalité importante, de grandes difficultés existent toujours pour subsister, vivre dignement, ou vivre bien, pour une part importante de la population, dans de nombreux pays en voie de développement, inégalité qui est en outre de plus en plus souvent considérée comme un problème en soi et non seulement comme un frein à un meilleur développement total.

Ainsi, une tension, ou un effet de renforcement, existe : développer l’éducation semble être une manière d’améliorer la productivité ou d’autres aspects du développement humain, mais il est possible que l’éducation profite inégalement à différentes parties de la population, ce qui pourrait avoir un effet positif ou négatif sur les inégalités. Cette revue de littérature s’intéresse pour ces raisons aux liens entre l’éducation et les inégalités.

Nous avons choisi de nous concentrer sur l’Amérique Latine, au sens large de vingt pays incluant l’Amérique Centrale, car cette zone dynamique en voie de développement comporte une certaine unité historique et de développement économique, mais également une diversité suffisante pour présenter potentiellement un panorama de situations et des résultats potentiellement contrastés. 

TODO définir ce que c’est l’éducation (capital humain,...) pour nous et les politiques éducatives aussi

TODO qlq lignes en plus pour contexte amérique latine depuis 80’s

\subsection{Sélection}

Une première recherche systématique \footnote{Toutes nos recherches sont effectuées par mots-clés dans EconLIT with Full Text et décrites précisément dans \cref{sec_search}.} focalisée sur les sujets éducation, inégalités et Amérique latine a renvoyé 156 résultats, devenus 21 articles et chapitres de livres après filtrage manuel. Cependant, après discussion, et constatant que seuls trois de ces articles étaient de source CNRS catégorie 1 ou 2 et la plupart contenait des observations générales (non spécifiques à certaines politiques éducatives), nous avons modifié notre stratégie.

Notre deuxième recherche systématique visait plus précisément à trouver des articles concernant des politiques éducatives spécifiques et établissant un lien entre l’éducation et les inégalités, et incluait les pays d’Amérique latine individuellement, pas seulement la zone entière. Celle-ci a renvoyé 56 résultats, dont restaient 23 articles après filtrage. Huit d’entre eux émanent d’une source de catégorie CNRS 1 ou 2. Nous avons cependant choisi d’en conserver également sept de catégorie CNRS 3 ou 4 \footnote{Ce choix est dû à une correspondance très bonne entre ces articles et notre sujet, et en vertu du fait que ces sources sont également \og{}des revues avec un processus d’arbitrage respectant les standards internationaux\fg{} qui peuvent contenir des articles accueillant des contributions originales sur des problématiques nationales (d’après la description par le CNRS de la catégorie 4).}. Nous avons donc analysé en détail 18 articles (qui ne sont pas tous résumés ici en détail faute de place). \footnote{Nous avons également recherché plus précisément des articles décrivant des essais randomisés, mais malgré plusieurs tentatives d’élargissement des contraintes et des mots-clés, n’avons réussi à trouver que trois articles avec les mots-clés retenus, dont aucun ne satisfaisait à notre sujet.}

\subsection{Sujet retenu}

Malgré l’élargissement de nos recherches, les articles concernant des politiques éducatives spécifiques restaient peu nombreux, nous avons donc choisi de considérer également les effets de l’éducation, en plus des effets de politiques éducatives. Concernant l’impact sur les inégalités, nous nous sommes concentrés sur les inégalités de revenu, mais nous avons également admis certains articles considérant non pas directement les inégalités de revenu mais des variables pouvant être raisonnablement considérées comme des proxys de l’inégalité de revenu, auquel cas notre analyse indique ce qu’il nous semble pouvoir être conclu de l’article concernant les inégalités de revenu.

\subsection{Plan}

\Cref{sec_educ} examine les enjeux de l’éducation dans les inégalités de revenus : à quels problèmes et défis les politiques éducatives doivent répondre ? Elle s’intéresse à l’impact de l’éducation sur les inégalités de revenu. \Cref{sec_struc} étudie les réformes structurelles, \cref{sec_transferts} les réformes de type transfert monétaire, et \cref{sec_svc} les réformes des services éducatifs. \Cref{sec_conc} conclut.

\section{Impact de l’éducation sur les inégalités de revenu}

\label{sec_educ}

\subsection{lien fondamental selon lequel les inégalités d’accès à l’éducation entrainent des inégalités de revenus}

\citet{psacharopoulos_poverty_1995} examine l’inégalité d’accès à l’éducation et ses liens avec les inégalités de revenus. Les auteurs observent une forte corrélation négative entre l’accomplissement d’années d’éducation et les inégalités de revenus des travailleurs \footnote{mesurées par le coefficient de Gini et le Theil Index} pour dix pays d’Amérique latine (dont le Brésil et le Costa Rica). Se tournant vers la causalité, ils observent que l’éducation est la variable dominante pour expliquer les inégalités : près du quart des inégalités de revenus peuvent être expliquées par l’appartenance à un certain niveau scolaire. De même, \citet{ferreira_rise_2008}, analysant les inégalités au Brésil de 1884 à 2004, déterminent le niveau d’éducation du chef de famille comme le facteur le plus important de l’inégalité globale. Ils relèvent toutefois que la causalité peut être inverse, les revenus familiaux pouvant eux même avoir un impact sur le niveau d'éducation reçu. Par ailleurs, \citet{ferreira_rise_2008} insiste sur les autres causes qui expliquent les inégalités de revenus : les variables de sexe, de race, d’âge, de zone d’habitat du chef de famille. \Citet{trejos_inequality_2004} également, analysant les années ’90, observe que l’éducation est la variable ayant le plus influencé les inégalités de revenus du travail dans la plupart des pays d’Amérique Latine.

Il semble donc consensuel qu’investir dans l’éducation contribue à réduire les inégalités. Mais deux phénomènes s’opposent : l’accès facilité à une éducation de qualité via une hausse de l’offre globale de compétences, qui diminue les inégalités, et les rendements croissants de l’éducation qui les augmentent.

\subsection{Inégalités d’accès}

Selon \citet{birdsall}, un cercle vicieux entraine une accumulation faible de capital humain en Amérique latine. Historiquement, un accès inégal des pauvres aux moyens de productions dont le capital humain (éducation) a diminué leur productivité. L’article conclut que les politiques éducatives n’ont pas bénéficié aux pauvres.

De manière assez similaire, les pays d’Amérique latine ont connu depuis plusieurs tendances en terme d'inégalités de revenus : de manière très synthétique entre les années 80 et 90 une hausse, puis entre 90 et 2000 une baisse des inégalités. Selon \citet{arabsheibani_2006}, entre 1988 et 1992, au contraire, des améliorations significatives des niveaux de capital humain ont eu lieu en Amérique latine, et celles-ci ont contribué à une réduction des inégalités de revenus car  la croissance réelle des revenus a été plus faible pour les personnes situées plus haut dans la distribution.

Concluant à une relation positive similaire, \citet{trejos_inequality_2004} mettent en avant la situation du Costa Rica comme pays d’Amérique latine ayant le moins d’inégalités (bien qu’ayant augmentées récemment). En constatant une distribution relativement égale de l’éducation et notamment des infrastructures entre villes et campagnes, ils suggèrent que les politiques publiques mises en place après les longues années de récession, qui universalisent l’éducation (notamment primaire), contribuent à baisser les inégalités de revenus. 

\citet{mejia_2024} constate que la hausse du nombre de ménages ayant accédé à l’enseignement secondaire n’a pas suffi à réduire les inégalités de revenus de manière substantielle. La persistance des écarts entre les groupes socio-économiques s’explique notamment par le faible accès à l’enseignement supérieur (près de \SI{10}{\percent}), qui reste dominé par les classes favorisées (\SI{11.6}{\percent} en 2014). \Citet{urbina_2018} également, dans son étude du programme mexicain \og{}Éducation pour tous\fg{} \footnote{\label{ft_mex} mis en place entre 1976 et 1992 afin de permettre \og{}l’expansion de l’éducation\fg{} (développement des infrastructures scolaires, formation des enseignants, fourniture de repas scolaires, etc.)}, observe que si la littérature conclut à une certaine efficacité sur la mobilité intergénérationnelle aux niveaux inférieurs de l’éducation, du primaire au secondaire inférieur (davantage d’enfants pauvres ont accéder à ces niveaux d’étude), le programme n’a pas surmonté les blocages dans l’enseignement supérieur.

Enfin, \citet{birdsall} mentionnent les obstacles à la poursuite du cursus pour les plus pauvres : problèmes financiers, d’accès physique, de qualité, déficits pour certains en arrivant à l’école, contrainte de travail…

Du côté des gouvernements, les politiques éducatives sont souvent insatisfaisantes pour encourager la hausse du supérieur. Analysant les \og{}national household survey\fg{} de 1993 et 1999, \citet{birdsall} ont constaté que les gouvernements d’Amérique Latine ont augmenté leurs dépenses en éducation et ont réussi à faire augmenter entre autres le taux de scolarisation des enfants, mais sans que cela n’affecte l’éducation supérieure. Contrairement à l’Asie de l’Est par exemple, l’Amérique latine n’a pas réussi à augmenter les taux dans le secondaire et le supérieur ; les ménages aisés, dont les enfants accèdent plus souvent à l’université, ont donc bénéficié plus fortement des subsides. \Cref{urbina} conclut également qu’une augmentation des dépenses en éducation peut se révéler insatisfaisant par insuffisance d’effet au niveau de l'enseignement supérieur.

\subsection{Rendement}

\label{sec_rendement}

De nombreux articles observent des rendements croissants de l’éducation dans les années 80/90, augmentant donc les inégalités. \Citet{arabsheibani_2006} s’intéresse à l’impact des rendements de l’éducation au Brésil, défini comme l’effet d’une année d’étude supplémentaire sur le salaire horaire. L’article observe, entre les années ’80 et ’90, une hausse de la qualification de la main d’œuvre compensée par celle des rendements de l’éducation, ce qui n’a pas permis une baisse des inégalités de revenus. L’estimation par un modèle de régression montre une augmentation de la part du nombre d’années d’études dans le salaire horaire, en comparaison du rendement de l’expérience \footnote{Après 16 années d'études, le rendement marginal moyen pour les hommes est de 28 à \SI{30}{\percent} en 1988, contre \SI{33}{\percent} en 1998, une année d’étude supplémentaire est donc davantage valorisée sur le marché du travail par une hausse de la rémunération. Les auteurs invitent toutefois à la prudence car le modèle peut être biaisé par l’impact des capacités individuelles non observées, et non celles acquises par l’école.}, liée d’après les auteurs à la libéralisation du pays couplée à l’essor des technologies. Par ailleurs, ce phénomène s’est fait au détriment du rendement de l’expérience. Les autres articles de notre étude confirment cette croissance des rendements de l’éducation \footnote{\og{}The subsequent reduction in income differential/returns to education (...) should contribute significantly to reductions in income disparities and poverty across the region\fg{} \citep{psacharopoulos_poverty_1995}; \citet{ferreira_rise_2008} prend également pour hypothèse que les rendements sont convexes ; \citet{carlson} mesure une relation forte entre le niveau scolaire atteint et les revenus du travail, concluant à de forts \og{}education premiums\fg{}, gains salariaux associés à une année supplémentaire d’éducation ou à l’obtention d’un niveau d’éducation, aux tranches supérieures d’éducation ; \citet{urbina} soutient l’hypothèse des rendements croissants de l’éducation, et affirme que ces années d’études génèrent une augmentation des salaires horaires espérés à la fin des études, or, si l’accès à l’enseignement supérieur n’est pas facilité aux plus pauvres les inégalités de revenus tendent à se maintenir.}. Dans les années 2000, \citet{levy} remarquent que les rendements de l’éducation dans la région ont diminué, ce qui a redirigé une partie des revenus du travail vers le bas de la distribution \footnote{cet effet a compté pour le tiers voire la moitié de la baisse des inégalités de revenus dans la région}. 

La compensation des deux phénomènes, amélioration de l’éducation et rendements croissants, se vérifie dans de nombreux pays en développement d’Amérique latine. \Citet{trejos_inequality_2004} indique qu’au Costa Rica, comparé au Salvador ou à d’autres pays avec de plus haut rendements mais plus inégaux, l’accès plus équitable à l’éducation et le rendement relativement bas \footnote{Une année de plus (notamment dans le secondaire et à l’université) se traduit par une relativement faible différence de revenus sur le marché du travail.} contribue à réduire les inégalités de revenus plus efficacement. Davantage de travailleurs y ont fréquenté l’université, bien que par rapport aux autres pays moins de travailleurs soient allés en primaire.

\subsection{Types d’établissements}

Pour conclure cette section, notons qu’il existe en Amérique latine de grandes disparités entre les établissements publics et privés, dont doivent tenir compte les décideurs politiques. \Citet{cavalcanti_2010} montrent que la part d’étudiants du public dans les filières compétitives est moins élevée que celle du privé, un constat qui se vérifie d’autant plus que les filières sont compétitives. Ils observent des barrières à l’entrée pour les élèves du public. \footnote{Le score moyen au test d’entrée à l’UFPE, une université publique élitiste brésilienne, est de 3.88 sur 10 pour les élèves du public, contre 4.63 pour ceux du privé.} Un modèle de régression montre une importance particulièrement élevée de la variable de provenance du lycée d’origine, et en particulier son statut public ou privé, par rapport par exemple au contexte familial \footnote{Les auteurs mettent en garde contre un biais via les efforts et capacités non observables.} : venir d’un lycée public a un impact négatif conséquent sur les performances scolaires. L’article suggère que le financement du secteur public, dans le but de préparer davantage les étudiants à l’entrée en études supérieures, est nécessaire à une plus grande égalité d’accès à l’éducation et donc une réduction des inégalités, une conclusion soutenue également par Bravo et Contreras (2001) footnote{\og{}The message for the present and the future is the need to improve the quality and equity of public education\fg{}} et par \citet{ferreira_rise_2008} \footnote{Pour ces auteurs, les politiques éducatives doivent améliorer le système éducatif public fréquenté majoritairement par des enfants issus de milieux pauvres afin de réduire les inégalités de revenus}.

La suite de cet article examine les modifications structurelles et locales ayant un impact sur l’éducation et les inégalités.

\section{Modifications structurelles pour promouvoir l’éducation}

\label{sec_struc}

\subsection{L’impact des réformes démocratiques sur l’enseignement et les inégalités}

Cette section examine l’effet d’une modification structurelle d’un pays vers plus de démocratie sur les inégalités, lorsque ces changements se produisent via une modification de l’éducation ou des politiques éducatives.
En particulier, on y examine les préférences des populations pour des investissements dans l’éducation.

Démocratie
Burs O1, Poverty and the Political Economy of Public Education Spending: Evidence from Brazil

L’auteur observe, avec d’autres, que l’éducation amène de hauts rendements pour les pauvres dans les pays en voie de développement (Psacharopoulos (1985, 1994) and Duflo (2001)), ce qui rend paradoxal le relatif manque d’investissement dans l’éducation, mais s’oppose à l’explication habituelle (Engerman and Sokoloff 2000; Mariscal and Sokoloff 2000; Acemoglu and Robinson 2001, 2006) qui en attribuent la responsabilité aux élites (ne souhaitant pas payer pour l’éducation des pauvres, souhaitant maximiser l’accès à une main d’œuvre pas chère ou éviter de donner du pouvoir aux pauvres). 
Analysant des données pré-existantes (Mulligan, Gil, and Sala-i-Martin (2004)), il observe une corrélation négative entre démocratie et investissement dans l’éducation dans les pays pauvres et positive dans les pays riches (toutes deux significatives au seuil 5% ou mieux). 
L’article soutient que l’investissement en éducation auprès des pauvres est plus faible quand la démocratie progresse simplement parce que les pauvres ne le souhaitent pas, lui préférant des bénéfices plus immédiats. 

Il fournit principalement trois arguments dans le contexte brésilien. L’un est observationel : les municipalités à faible revenu médian sont moins favorisées dans les votes, alors que celles à haut revenu médian sont plus favorisées, quand leur action a accru les dépenses d’éducation. 
Un autre est expérimental : les préférences des sujets sont récoltées après les avoir informé que le gouvernement a effectué des dépenses d’éducation au détriment de transferts d’argent ; les pauvres indiquent une appréciation moins favorable comparé à pas d’information, et les riches, plus favorable (effet statistiquement significatif si on considère les quartiles extrêmes). 
Le troisième, expérimental également, accroit le revenu de certains sujets, et observe que cela les conduit à privilégier plus fortement l’éducation (sous forme de tutoring après l’école) que l’argent immédiat.
L’auteur lie son hypothèse aux études qui indiquent une plus forte préférence pour le présent chez les pauvres, via une utilité de l’argent (y compris sous forme de contraintes fortes) plus grande ou via de possibles biais cognitifs tels que mis en évidence par Kahneman & Tversky, et aux études liant démocratie et éducation, bien que ce dernier point soit contrasté dans la littérature.

Réformes agraires
Albertus O2, Land Reform and Human Capital Development: Evidence from Peru

Les auteurs partent du constat consensuel que les grands propriétaires terriens ont souvent bloqué l’éducation des plus pauvres, de l’ère coloniale au milieu du XXème siècle, ce qui conduit souvent à apprécier les réformes de la propriété de la terre pour réduire les inégalités.
La littérature a souvent examiné l’impact de l’inégalité de propriété terrienne sur l’accumulation de capital humain, constatant souvent une amélioration de l’offre d’éducation avec la diminution de la concentration de propriété.
Ils s’intéressent ici à un test empirique direct d’une réforme de propriété de la terre sur le capital humain, grâce à des données locales sur les terrains transférés et l’éducation individuelle (Census data et surveys sur l’éducation) suite à une réforme menée par les militaires au Pérou de 1968 à 1980, afin d’évaluer les effets à long terme des réformes de terrain.
La moitié des terres privées fut mise en système coopératif (pas de propriété privée, pas de transaction sur marché), dans une diversité de lieux, d’intensité, de temporalité.

Les auteurs constatent, de manière surprenante, que les réformes ont baissé l’accumulation de capital relative, les exposés ayant perdu un tiers d’année d’éducation en moyenne.
Ils postulent que le mécanisme est une moindre demande d’éducation : l’opportunité de travail à la terre mène au piège du travail de la terre, à la tentation de quitter l’école.
Ceci ne contredit pas l’accroissement généralement considéré de l’offre, mais va dans un sens différent, ce qui peut d’après les auteurs expliquer les différences de relations entre inégalité de terre et éducation.
Migration vers la ville mène à plus d’opportunités d’élévation sociale.
Bien que l’article d’Albertus (2020) ne référence pas Burs, qui lui est antérieur, nous constatons avec intérêt une convergence de leurs conclusions concernant un effet négatif sur l’éducation d’une préférence accrue des gens plus pauvres pour les revenus immédiats.
Ces deux articles suggèrent que, paradoxalement, l’augmentation de la démocratie pourrait exacerber les inégalités, si l’on considère ce qui résulte de l’éducation, en permettant aux plus pauvres de faire valoir leurs préférences pour une redistribution de bénéfices immédiats plutôt que plus d’investissement dans l’éducation.
Bien sûr, cette conclusion est à prendre avec beaucoup de prudence, considérant le faible nombre d’études analysées ici sur cette question précise, et ne dit rien de l’effet total sur les inégalités (outre l’effet via l’éducation), puisque la redistribution d’argent modifie bien sûr les inégalités.

Littérature
Pouvoir politique concentré (& ressources concentrées) => institutions excluantes & développement plus lent.
La littérature suggère dû historiquement à distribution terres inégales par limitation de l’offre : grands propriétaires empêchent écoles, restreignent éducation, extraient le surplus des travailleurs au lieu d’investir en capital humain ; lié à ralentissement de l’offre d’institutions d’éducation.
On a donc vu des réformes de terres et d’autres sont prévues.

Libéralisation économique

L’article d’Arabsheibani et al. (2006) révèle également comment le contexte économique et commercial influence les rendements du travail qualifié.
Cet article montre que, contrairement à la théorie du commerce international, qui prédit une spécialisation du Brésil dans le travail non qualifié, l’ouverture commerciale des années 80, marquée par une forte baisse des tarifs douaniers, a augmenté la demande de travailleurs qualifiés en raison des besoins technologiques supplémentaires.
Cela a profité aux personnes ayant suivi un enseignement secondaire, augmentant ainsi les rendements de l’éducation et, par conséquent, les inégalités.
À l’inverse, l’article de Ferreira prédit l’effet opposé de la libéralisation.
Selon les auteurs de cet article, la libéralisation du commerce a favorisé la croissance d’un secteur moderne dans l’agriculture au Brésil, entraînant une convergence des revenus ruraux et urbains ainsi qu’une baisse des inégalités.

\section{Transferts monétaires}

\label{sec_transferts}

Nous tournant maintenant vers des politiques plus ciblées, nous observons que les articles de recherche abordent de nombreuses politiques éducatives basées sur des transferts monétaires.
Deux d’entre elles semblent particulièrement intéressantes au vu de leur impact sur les inégalités de revenus : le système de cash transfer et le système de bons scolaires.

Depuis la fin des années 90, l’Amérique Latine a mis en place des politiques d’éducation de cash transfer, connu comme l’un des programmes qui redistribue le plus les revenus vers les plus pauvres (PROGRESA au Mexique ou encore PANES en Urugay).
Selon les auteurs, près d’un quart des ménages pauvres en Amérique Latine en bénéficient.
Certains d’entre eux, mais pas tous, prennent la forme de cash transfer conditionnés c’est-à-dire que les ménages qui en bénéficient doivent s’assurer que leurs enfants en âge d’aller à l’école soient scolarisés, et aillent régulièrement chez le médecin.
L’éligibilité est souvent déterminée par des caractéristiques composites des ménages comme les actifs détenus, l’accès à certains services sociaux, et sont corrélés à la consommation ou aux revenus.


Les auteurs concluent, à travers des évaluations randomisées, que cette mesure a augmenté le taux de scolarisation, le taux de présence à l’école et baissé de manière substantielle la pauvreté et les inégalités de revenus entre les plus aisés et les moins aisés.
Au Mexique par exemple, trois ans de programmes de cash transfer ont environ amené à 0.3 année en plus complétée par enfant, si l’on compare à une situation sans cash transfer.
Ces effets sur la baisse des inégalités de revenus entre classe restent tout de même limités.
Les investissements dans ce programme sont fortement variables par pays : entre 2006 et 2013 celui de l’Équateur a plus que doublé tandis que celui de la Colombie baissait.
D’après leurs calculs, les programmes reposant sur des cash transfer ont participé à hauteur d’un quart à la baisse des inégalités de revenus.

Trois problèmes subsistent quant à l’implantation du système de cash transfer et qui impactent les inégalités de revenus.
La hausse du taux de scolarisation ne s’est pas nécessairement traduite par de meilleurs résultats d’apprentissage pour les enfants ayant bénéficié du programme. \Citet{behram_2009} ont remarqué que les enfants mexicains ayant reçu les cash transfer n’avaient pas de meilleurs résultats scolaires que ceux qui n’en n’avaient pas bénéficié. Ils avancent que les enfants venant d’un milieu rural ou modeste, bien que pouvant réaliser davantage d’années d’études, arrivent à l’école avec des déficits en termes de développement et de santé, de faibles compétences cognitives et un retard de croissance en termes de niveau attendu. Bien qu’ils soient scolarisés, leur retard accumulé et la qualité des écoles qu’ils fréquentent les envoient sur le marché du travail avec des désavantages en termes de productivité, impliquant un niveau de salaire bas. De plus, ces programmes ont un effet négatif sur les incitations à travailler. Les cash transfer occupent une part importante du revenu des ménages en bénéficiant, ce qui peut les inciter à rester ou du moins apparaitre pauvres, ce qui tend de surcroit à favoriser l’emploi informel (pour éviter la déclaration des revenus). Par exemple, \citet{amarante_2011} expliquent que le programme PANES en Uruguay a baissé l’emploi formel parmi les hommes à la suite de l’implantation de ces mesures. 

\Citet{bravo_} étudient quand à eux l’impact de système de bons scolaires \footnote{Plus précisément, l’introduction du programme national de bons scolaires (\og{}school voucher\fg{}) au Chili à partir de 1981.} sur plusieurs variables dont les inégalités de revenus (qui nous intéressent ici). Il existe trois types d’écoles au Chili : les écoles dites municipales, les écoles privées subventionnées et les écoles privées non subventionnées. Pour mettre les écoles en concurrence, le gouvernement Pinochet a octroyé un montant fixe par élève, laissant aux parents le choix du type d’école. Grâce à des données longitudinales et d'enquêtes auprès des ménages sur les individus ayant obtenu leur éducation avant, pendant et après la réforme, les auteurs ont étudié les changements consécutifs à son introduction. La baisse des couts de scolarité, l’introduction des bons scolaires et les changements dans les revenus de la scolarité ont augmenté le taux de scolarisation, surtout des enfants vivant dans les milieux ruraux et surtout des écoles privées : le bénéfice de la réforme tout au long de la scolarité est de \SI{0.6}{\percent}, \SI{3.6}{\percent} et \SI{1.8}{\percent} sur les taux d’achèvement des études primaires, secondaires et supérieures respectivement, et de \SI{3.1}{\percent} sur le taux de fréquentation universitaire. Les auteurs constatent que la réforme n’a pas conduit à une augmentation globale des revenus moyens gagnés par les étudiants sortant d’études : les avantages en revenus dûs à un niveau supérieur d’éducation sont compensés par l’entrée sur le marché du travail plus tardive (due à l’allongement des études) et par la diminution après la réforme du rendement de l’enseignement secondaire. Cependant, en analysant la distribution des revenus, les auteurs montrent une augmentation des revenus aux centiles inférieurs de la distribution et une diminution aux centiles supérieurs, générant ainsi une diminution modeste des inégalités de revenus. Enfin, ils concluent que cette réforme a aussi bien profité aux ménages les plus aisés qu'aux ménages les moins aisés. 

\section{Services éducatifs}

\label{sec_svc}

Nous examinons ici les politiques éducatives qui ne consistent pas en des transferts monétaires directs mais sont susceptibles d’affecter la répartition des revenus en Amérique latine. 

\subsection{Effet de garderie}

Il est consensuel que les réformes permettant de laisser les enfants à l’école sur une plus longue durée hebdomadaire facilitent le travail des femmes. \Citet{berthelon} s’intéressent à une réforme au Chili qui a allongé les heures d’école, et étudient son effet sur l’emploi des femmes, en fonction de leur niveau d’éducation préalable. Ils observent des effets plus bénéfiques pour les mères avec une éducation plus faible, et suggèrent de diriger spécifiquement les plus longues périodes scolaires vers les populations plus vulnérables afin de permettre aux femmes d’accéder à plus de revenus. Dans une étude comparative, \citet{staab_2011} s’intéressent à l’impact de deux politiques visant à étendre l’accès aux services de garde et d’éducation au Mexique et au Chili \footnote{Respectivement  Programa Guarderías y Estancias Infantiles et Chile Crece Contigo}, ce dernier également étudié par \citet{berthelon}, et mettent en avant que les deux politiques d’éducation réduisent la pauvreté et la dépendance, mais par des moyens et avec une efficacité différents. Le programme chilien, centré sur l’éducation et l’égalité des chances, vise à réduire les disparités intergénérationnelles à long terme. Il fournit davantage de revenus aux femmes et des services institutionnels destinés aux enfants des familles à faible revenu (les trois derniers quintiles), leur permettant d’aller plus loin dans leurs études et de bénéficier de revenus élevés. En revanche, le programme mexicain se concentre principalement sur l’accès des mères travaillant dans le secteur informel à l’emploi, avec un système de garderies subventionnées à domicile ou communautaires qui dispensent des soins et une éducation. Toutefois, l’entrée du personnel n’est pas conditionnée à un certain niveau de diplôme, et la qualité des soins et de l’éducation est souvent médiocre, exacerbant ainsi les inégalités de revenus entre les classes sociales.

L’article discute de deux limites à ce type de politiques éducatives : un manque de moyens financiers \footnote{Étendre ces programmes à plus d’une demi-journée et assurer l’emploi de personnel qualifié permettrait une mise en place de politiques éducatives de qualité et donc une approche intégrée de ces deux objectifs.}, et les traditions de genre encore fortement ancrées dans les deux pays \footnote{Les idéaux motheralists, souhaitant que les mères s’occupent des enfants au foyer, particulièrement au Mexique, freinent le financement de telles politiques.}. En somme, ces politiques éducatives sont plus efficaces lorsqu’elles ciblent les populations les plus pauvres, mais des politiques universelles permettent un meilleur consentement à l’impôt et augmentent les recettes ce qui peut pallier le manque de moyens financiers.

\Citet{staab_} concluent que des politiques finançant des institutions éducatives ont un impact direct sur l’égalité des chances permettant à tout enfant indépendamment de son milieu social d’origine d’accéder à une éducation longue et qualitative favorisant l’obtention d’un emploi qualifié. Celles-ci feraient donc baisser par conséquent les inégalités de revenus à long terme. 

\subsection{Redoublement}

Les politiques éducatives liées au redoublement affectent également les inégalités de revenu. \Citet{resende} s’intéressent à l’abandon des études, qui est généralement reconnu comme étant corrélé positivement aux désavantages économiques. Le choix du système de promotion d’une année à l’autre (auto-promotion, passer directement au niveau supérieur indépendamment de ses résultats scolaires, versus redoublement si résultats insuffisants) influence le taux d’abandon. Le redoublement est souvent dans la littérature associée à plus d’abandon, mais pas toujours, les résultats semblant très contextuels. Les auteurs constatent qu’au Brésil \footnote{Le Brésil a la plus faible part d’adultes ayant atteint l’éducation secondaire parmi les pays de l’OCDE et d’Amérique latine.}, le redoublement a fortement diminué, sans constater pourtant un moindre abandon. Les auteurs notent que la question se pose cependant du lien entre la stratégie éducative et ce résultat, ainsi que de l’éventuelle différence d’effet selon la richesse des étudiants. Ceci motive leur étude au Brésil qui analyse l’abandon des étudiants selon leur richesse en comparant le système d’auto-promotion et redoublement. Leurs modèles \footnote{Ils prônent l’usage d’un modèle probit à effets aléatoires (Random Effects) pour prendre en compte l’hétérogénéité des populations riches et pauvres.} suggèrent une hausse de la probabilité d’abandon de l’école en cas de redoublement (par rapport à une stratégie d’auto-promotion), pour les populations riches et pauvres, mais plus particulièrement marquée pour cette dernière. Ils concluent qu’une stratégie d’auto-promotion pourrait diminuer l’abandon, et réduire l’écart entre les enfants à hauts et faibles revenus. La force de ces conclusions nous semble à nuancer car les données analysées ne considèrent pas la différence entre deux stratégies éducatives, les observations concernent seulement le redoublement effectif des étudiants ou leur réussite (dans un cadre avec stratégie éducative fixée) ; les modèles traitent donc de données seulement indirectement liées aux conclusions obtenues.

\section{Conclusion}

\label{sec_conc}

On en conclut que des politiques éducatives combinant un accès universel et qualité éducative fait davantage baisser les inégalités de revenus. De plus, viser les zones rurales et les populations marginalisés (les plus pauvres, les femmes) via des réformes éducatives leur donnent la possibilité d’améliorer leurs revenus, et par conséquent diminuer les inégalités.

résumé + au moins 1 manque de la littérature

\appendix
\section{Sources}
\label{sec_search}
Nous avons choisi de nous appuyer sur la base de données de EBSCOhost \emph{EconLIT with Full Text} qui semble présenter un bon compromis entre sa couverture large (nombreuses sources en économie) et sa spécialisation (réduisant les risques de trop nombreux résultats non pertinents). En outre, elle permet une recherche par mots clés avancée.
Voici comment l’éditeur la décrit \citep{ebscohost_econlit_2024}.

“EconLit, the American Economic Association's electronic database, is the world's foremost source of references to economic literature. EconLit adheres to the high quality standards long recognized by subscribers to the Journal of Economic Literature (JEL) and is a reliable source of citations and abstracts to economic research dating back to 1886. It provides links to full-text articles in all fields of economics, including capital markets, country studies, econometrics, economic forecasting, environmental economics, government regulations, labor economics, monetary theory, urban economics, and much more. EconLit uses the JEL classification system and controlled vocabulary of keywords to index six types of records: journal articles, books, collective volume articles, dissertations, working papers, and full-text book reviews from the Journal of Economic Literature. These sources bring the total records available in the database to more than 1.2 million.

EconLit with Full Text contains all of the indexing available in EconLit, plus full text for nearly 600 journals, including the American Economic Association journals with no embargo (American Economic Review, Journal of Economic Literature, Journal of Economic Perspectives, and the four American Economic Journal titles). This database also contains many non-English full-text journals in economics and finance. Volume and issue browsing is available for all full-text journals.”

\section{Couverture et mots-clés cherchés}
Notre recherche s’intéresse aux articles couvrant les éléments suivants, appelés ici \og{}sujets\fg{} : l’éducation, les inégalités, la liaison entre ces deux éléments, et l’Amérique latine. La liaison représente le lien de cause à effet, la corrélation, ou autres relations liant ces deux éléments. Pour chacun de ces éléments, nous réunissons des synonymes ou des termes pertinents, y compris des variations de suffixes. Voici les termes retenus.

\begin{description}
  \item[Éducation] Le mot clé retenu est \texttt{education*}, pour inclure l’adjectif \emph{educational}.
  \item[Inégalités] Nous avons retenus les mots clés \texttt{equality}, \texttt{inequality}, \texttt{egalitar*} (couvrant \emph{egalitarian}, \emph{egalitarianism}, …) et \texttt{inegalitar*}. Notons que le moteur de \citet{ebscohost_searching_2024} permet d’après sa documentation d’utiliser \texttt{\#\#equality} pour couvrir \emph{equality} et \emph{inequality}, mais nous avons constaté de nombreux bugs dans cet usage.
  \item[Liaison] Nous avons adopté les mots clés \texttt{caus*}, \texttt{impact*}, \texttt{increase*}, \texttt{decrease*}, \texttt{raise*}, \texttt{lower*}, \texttt{influence*}, \texttt{relation*} et \texttt{correlation*}.
  \item[Politique] Nous avons cherché les mots clés \texttt{policy}, \texttt{policies} et \texttt{reform*} pour cibler plus spécifiquement les politiques éducatives.
  \item[Polique éducative] Nous avons cherché les mots clés \texttt{education* N5 policy}, \texttt{education* N5 policies} et \texttt{education* N5 reform*} pour cibler différemment les politiques éducatives. 
  \item[Amérique latine] Nous avons utilisé \texttt{Latin America*}, qui couvre \emph{Latin America} et \emph{Latin American} (apparaissant par exemple dans \emph{Latin American Journal of Economics}). 
  \item[AL5] L’Amérique latine et les cinq pays les plus peuplés. Concernant les pays, nous avons également inclus les gentilés, par exemple avec \texttt{Mexic*} pour inclure \emph{Mexico} et \emph{Mexican}.
  \item[AL20] L’Amérique latine et tous les pays la composant.
  \item[RCT] Nous avons utilisé \texttt{randomi*} qui inclut \emph{randomized}, \emph{randomised}, et \texttt{RCT}.
\end{description}
De plus, nous avons examiné les concepts de la classification JEL en y cherchant les mots-clés et variantes indiquées ici. Le concept \texttt{I24}, \emph{Éducation et inégalités}, correspond à notre intérêt.
Nous avons donc façonné la requête de manière à trouver également les articles appartenant à cette catégorie même si les mots-clés concernant l’éducation et les inégalités ne sont pas trouvés.

Nous avons cherché les mots clés correspondants dans le titre de l’article, les mots-clés associés à l’article, le résumé et la source de l’article. Ce dernier point est dû à la possibilité qu’un article publié dans Latin American Journal of Economics, par exemple, couvre l’Amérique latine sans que ce fait soit mentionné par les auteurs explicitement.

Voici des exemples de morceaux de requête obtenus, en commençant par celle concernant l’éducation:
\begin{equation}
  \label{eq:education}
  \left.\begin{aligned}
    &\texttt{TI education* OR}\\ 
    &\texttt{SO education* OR}\\ 
    &\texttt{KW education* OR}\\ 
    &\texttt{AB education*}.
  \end{aligned}\right.
\end{equation}
Les morceaux concernant les autres sujets suivent une logique similaire:
\begin{equation}
  \label{eq:equality}
  \left.\begin{aligned}
    &\texttt{TI (equality OR inequality OR egalitar* OR inegalitar*) OR}\\ 
    &\texttt{SO (equality OR inequality OR egalitar* OR inegalitar*) OR}\\ 
    &\texttt{KW (equality OR inequality OR egalitar* OR inegalitar*) OR}\\ 
    &\texttt{AB (equality OR inequality OR egalitar* OR inegalitar*)};
  \end{aligned}\right.
\end{equation}
\begin{equation}
  \label{eq:latin}
  \left.\begin{aligned}
    &\texttt{TI "Latin America*" OR}\\ 
    &\texttt{SO "Latin America*" OR}\\ 
    &\texttt{KW "Latin America*" OR}\\ 
    &\texttt{AB "Latin America*"}.
  \end{aligned}\right.
\end{equation}

Enfin, nous avons ensuite filtré les résultats pour ne conserver que les sources de type \emph{Academic Journal} et parfois \emph{Collective Volume Articles}. Ici encore, une directive permet en principe de les filtrer en rédigeant la requête (\texttt{PT "Journal Article" OR PT "Collective Volume Article"}), ou de conserver uniquement les articles avec revue par les pairs (\texttt{RV 1}) mais nous avons constaté des bugs dans ces deux fonctionnalités et avons privilégié l’usage de l’interface graphique prévue à cet effet.

Définissons la sous-requête \emph{ÉducInég} comme suit.
\begin{equation}
  \label{eq:query}
  \left.\begin{aligned}
    &\Bigl(\bigl(\eqref{eq:education} \text{ AND } \eqref{eq:equality}\bigr) \text{ OR CC I24}\Bigr)\text{ AND LA English}.
  \end{aligned}\right.
\end{equation}

Quelques résultats sont donnés ci-dessous
(liens passant par le proxy de Dauphine, il faut ensuite restreindre les types de sources).
Chacune est décomposée pour des raisons de présentation mais bien sûr exécutée en une seule requête sur la base de données en remplaçant les références par les morceaux de requêtes entre parenthèses. Par exemple, \emph{ÉducInég ∩ AL} représente la requête résumée  $\Bigl(\bigl(\eqref{eq:education} \text{ AND } \eqref{eq:equality}\bigr) \text{ OR CC I24}\Bigr)\text{ AND LA English AND }\eqref{eq:latin}$, soit : 
\begin{center}
\texttt{$\Bigl($(TI education* OR SO education* OR KW education* OR AB education*) AND $\bigl($(TI (equality OR inequality OR egalitar* OR inegalitar*) OR SO (equality OR inequality OR egalitar* OR inegalitar*) OR KW (equality OR inequality OR egalitar* OR inegalitar*) OR AB (equality OR inequality OR egalitar* OR inegalitar*)$\bigr)$ OR CC I24$\Bigr)$ AND (TI "Latin America*" OR SO "Latin America*" OR KW "Latin America*" OR AB "Latin America*")}.
\end{center}
Le A en suffixe désigne le nombre d’articles en comptant uniquement les articles de journaux académiques.

\begin{description}
  \item[\href{https://search-ebscohost-com.proxy.bu.dauphine.fr/login.aspx?direct=true&db=eoh&bquery=((((TI+(equality+OR+inequality+OR+egalitar*+OR+inegalitar*)+OR+SO+(equality+OR+inequality+OR+egalitar*+OR+inegalitar*)+OR+KW+(equality+OR+inequality+OR+egalitar*+OR+inegalitar*)+OR+AB+(equality+OR+inequality+OR+egalitar*+OR+inegalitar*))+AND+(TI+education*+OR+SO+education*+OR+KW+education*+OR+AB+education*))+OR+CC+I24)+AND+(TI+(caus*+OR+impact*+OR+increase*+OR+decrease*+OR+raise*+OR+lower*+OR+influence*+OR+relation*+OR+correlation*)+OR+SO+(caus*+OR+impact*+OR+increase*+OR+decrease*+OR+raise*+OR+lower*+OR+influence*+OR+relation*+OR+correlation*)+OR+KW+(caus*+OR+impact*+OR+increase*+OR+decrease*+OR+raise*+OR+lower*+OR+influence*+OR+relation*+OR+correlation*)+OR+AB+(caus*+OR+impact*+OR+increase*+OR+decrease*+OR+raise*+OR+lower*+OR+influence*+OR+relation*+OR+correlation*))+AND+((TI+("Latin+America*"+OR+Brazil*+OR+Mexic*+OR+Columbia*+OR+Argentin*+OR+Peru*+OR+Venezuela*+OR+Chile*+OR+Guatemala*+OR+Ecuador*+OR+Bolivia*+OR+Cuba*+OR+Dominican+OR+Hondura*+OR+Paraguay*+OR+Salvador*+OR+Nicaragua*+OR+Costa+OR+Panama*+OR+Uruguay*+OR+Puerto))+OR+(SO+("Latin+America*"+OR+Brazil*+OR+Mexic*+OR+Columbia*+OR+Argentin*+OR+Peru*+OR+Venezuela*+OR+Chile*+OR+Guatemala*+OR+Ecuador*+OR+Bolivia*+OR+Cuba*+OR+Dominican+OR+Hondura*+OR+Paraguay*+OR+Salvador*+OR+Nicaragua*+OR+Costa+OR+Panama*+OR+Uruguay*+OR+Puerto))+OR+(KW+("Latin+America*"+OR+Brazil*+OR+Mexic*+OR+Columbia*+OR+Argentin*+OR+Peru*+OR+Venezuela*+OR+Chile*+OR+Guatemala*+OR+Ecuador*+OR+Bolivia*+OR+Cuba*+OR+Dominican+OR+Hondura*+OR+Paraguay*+OR+Salvador*+OR+Nicaragua*+OR+Costa+OR+Panama*+OR+Uruguay*+OR+Puerto))+OR+(AB+("Latin+America*"+OR+Brazil*+OR+Mexic*+OR+Columbia*+OR+Argentin*+OR+Peru*+OR+Venezuela*+OR+Chile*+OR+Guatemala*+OR+Ecuador*+OR+Bolivia*+OR+Cuba*+OR+Dominican+OR+Hondura*+OR+Paraguay*+OR+Salvador*+OR+Nicaragua*+OR+Costa+OR+Panama*+OR+Uruguay*+OR+Puerto)))+AND+LA+English+AND+((education*+N5+policy)+OR+(education*+N5+policies)+OR+(education*+N5+reform*)))}{ÉducInég ∩ PolÉdu ∩ Liaison ∩ AL-20}] 64 A
  \item[\href{https://search-ebscohost-com.proxy.bu.dauphine.fr/login.aspx?direct=true&db=eoh&bquery=(((TI+(equality+OR+inequality+OR+egalitar*+OR+inegalitar*)+OR+SO+(equality+OR+inequality+OR+egalitar*+OR+inegalitar*)+OR+KW+(equality+OR+inequality+OR+egalitar*+OR+inegalitar*)+OR+AB+(equality+OR+inequality+OR+egalitar*+OR+inegalitar*))+AND+(TI+education+OR+SO+education+OR+KW+education+OR+AB+education))+OR+CC+I24)+AND+(TI+"Latin+America*"+OR+SO+"Latin+America*"+OR+KW+"Latin+America*"+OR+AB+"Latin+America*")}{Originale}] 156 articles.
  \item[ÉducInég ∩ Pol ∩ Liaison ∩ AL-5] 103 A
  \item[ÉducInég ∩ Pol ∩ PolÉdu ∩ Liaison ∩ AL-5] 33 A
  \item[ÉducInég ∩ PolÉdu ∩ Liaison ∩ AL-5] 47 A
  \item[ÉducInég ∩ Pol ∩ Liaison ∩ AL-20] 125 A
  \item[ÉducInég ∩ Pol ∩ PolÉdu ∩ Liaison ∩ AL-20] 43 A
  \item [\href{https://search-ebscohost-com.proxy.bu.dauphine.fr/login.aspx?direct=true&db=eoh&bquery=(((TI+(equality+OR+inequality+OR+egalitar*+OR+inegalitar*)+OR+SO+(equality+OR+inequality+OR+egalitar*+OR+inegalitar*)+OR+KW+(equality+OR+inequality+OR+egalitar*+OR+inegalitar*)+OR+AB+(equality+OR+inequality+OR+egalitar*+OR+inegalitar*))+AND+(TI+education+OR+SO+education+OR+KW+education+OR+AB+education))+OR+CC+I24)+AND+((TI+("Latin+America*"+OR+Brazil*+OR+Mexic*+OR+Columbia*+OR+Argentin*+OR+Peru*+OR+Venezuela*+OR+Chile*+OR+Guatemala*+OR+Ecuador*+OR+Bolivia*+OR+Cuba*+OR+Dominican+OR+Hondura*+OR+Paraguay*+OR+Salvador*+OR+Nicaragua*+OR+Costa+OR+Panama*+OR+Uruguay*+OR+Puerto))+OR+(SO+("Latin+America*"+OR+Brazil*+OR+Mexic*+OR+Columbia*+OR+Argentin*+OR+Peru*+OR+Venezuela*+OR+Chile*+OR+Guatemala*+OR+Ecuador*+OR+Bolivia*+OR+Cuba*+OR+Dominican+OR+Hondura*+OR+Paraguay*+OR+Salvador*+OR+Nicaragua*+OR+Costa+OR+Panama*+OR+Uruguay*+OR+Puerto))+OR+(KW+("Latin+America*"+OR+Brazil*+OR+Mexic*+OR+Columbia*+OR+Argentin*+OR+Peru*+OR+Venezuela*+OR+Chile*+OR+Guatemala*+OR+Ecuador*+OR+Bolivia*+OR+Cuba*+OR+Dominican+OR+Hondura*+OR+Paraguay*+OR+Salvador*+OR+Nicaragua*+OR+Costa+OR+Panama*+OR+Uruguay*+OR+Puerto))+OR+(AB+("Latin+America*"+OR+Brazil*+OR+Mexic*+OR+Columbia*+OR+Argentin*+OR+Peru*+OR+Venezuela*+OR+Chile*+OR+Guatemala*+OR+Ecuador*+OR+Bolivia*+OR+Cuba*+OR+Dominican+OR+Hondura*+OR+Paraguay*+OR+Salvador*+OR+Nicaragua*+OR+Costa+OR+Panama*+OR+Uruguay*+OR+Puerto)))+AND+(TI+(randomi*+OR+RCT)+OR+SO+(randomi*+OR+RCT)+OR+KW+(randomi*+OR+RCT)+OR+AB+(randomi*+OR+RCT))+AND+LA+English}{ÉducInég ∩ AL-20 ∩ RCT}] 3 A
\end{description}

Voici les articles retenus: \citet{lopez-acevedo_mexico_2004, gonzalez_overcoming_2012, psacharopoulos_poverty_1995, staab_putting_2011, esquivel_dynamics_2011, braunstein_impact_2018, lustig_impact_2016, messina_twenty_2020, contreras_wage_2011, patrinos_note_2009, legovini_can_2005, battiston_could_2014, amarante_decomposing_2016, carlson_education_2002, birdsall_education_1998, birdsall_education_1997, astorquiza_bustos_income_2021, lustig_income_2014, duryea_labor_2000}

% \bibliography{zotero, zotlit}
\bibliography{manual, sel}

\end{document}
